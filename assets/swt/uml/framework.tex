\documentclass{article}
\usepackage[utf8]{inputenc} % use utf8 file encoding for TeX sources
\usepackage[T1]{fontenc}    % avoid garbled Unicode text in pdf
\usepackage{tikz}
\usepackage{tikz-uml}
\usepackage{amssymb}
\usetikzlibrary{automata, positioning, arrows}

\newcommand{\code}{\texttt}

\begin{document}
\pagestyle{empty}
\begin{figure}[ht]
\centering

\begin{tikzpicture}

  \tikzumlset{
    fill class=none,
    fill note=none,
    fill package=none
  }

  \begin{umlpackage}{Rahmenarchitektur}

  \umlclass{Anwendung}{}{main()}

  \umlclass[right=2cm of Anwendung]{Grafik}{}{
    zeichne() \\
    loesche()
  }

  \umlclass[below=2cm of Grafik]{Rechteck}{}{
    zeichne() \\
    loesche()
  }

  \umlclass[left=2cm of Rechteck]{Linie}{}{
    zeichne() \\
    loesche()
  }

  \umlclass[right=2cm of Rechteck]{Elipse}{}{
    zeichne() \\
    loesche()
  }

  \end{umlpackage}

  \begin{umlpackage}{Erweiterung}

  \umlclass[below=2cm of Rechteck]{Quadrat}{}{
    zeichne() \\
    loesche()
  }

  \umlclass[right=5cm of Quadrat]{Ikone}{}{
    zeichne() \\
    loesche()
  }

  \umlnote[left=2cm of Quadrat]{Quadrat}{\textbf{Problem}: Wie macht man der Rahmenarchitektur nun klar das die Einschübe existieren?}

  \end{umlpackage}

  \umlassoc{Anwendung}{Grafik}

  \umlinherit[geometry=|-|]{Linie}{Grafik}
  \umlinherit[geometry=|-|]{Rechteck}{Grafik}
  \umlinherit[geometry=|-|]{Elipse}{Grafik}

  \umlinherit{Quadrat}{Rechteck}
  \umlinherit[geometry=|-]{Ikone}{Grafik}

\end{tikzpicture}
\end{figure}
\end{document}
